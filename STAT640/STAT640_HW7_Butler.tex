% Options for packages loaded elsewhere
\PassOptionsToPackage{unicode}{hyperref}
\PassOptionsToPackage{hyphens}{url}
%
\documentclass[
]{article}
\usepackage{amsmath,amssymb}
\usepackage{lmodern}
\usepackage{iftex}
\ifPDFTeX
  \usepackage[T1]{fontenc}
  \usepackage[utf8]{inputenc}
  \usepackage{textcomp} % provide euro and other symbols
\else % if luatex or xetex
  \usepackage{unicode-math}
  \defaultfontfeatures{Scale=MatchLowercase}
  \defaultfontfeatures[\rmfamily]{Ligatures=TeX,Scale=1}
\fi
% Use upquote if available, for straight quotes in verbatim environments
\IfFileExists{upquote.sty}{\usepackage{upquote}}{}
\IfFileExists{microtype.sty}{% use microtype if available
  \usepackage[]{microtype}
  \UseMicrotypeSet[protrusion]{basicmath} % disable protrusion for tt fonts
}{}
\makeatletter
\@ifundefined{KOMAClassName}{% if non-KOMA class
  \IfFileExists{parskip.sty}{%
    \usepackage{parskip}
  }{% else
    \setlength{\parindent}{0pt}
    \setlength{\parskip}{6pt plus 2pt minus 1pt}}
}{% if KOMA class
  \KOMAoptions{parskip=half}}
\makeatother
\usepackage{xcolor}
\IfFileExists{xurl.sty}{\usepackage{xurl}}{} % add URL line breaks if available
\IfFileExists{bookmark.sty}{\usepackage{bookmark}}{\usepackage{hyperref}}
\hypersetup{
  hidelinks,
  pdfcreator={LaTeX via pandoc}}
\urlstyle{same} % disable monospaced font for URLs
\usepackage[margin=1in]{geometry}
\usepackage{color}
\usepackage{fancyvrb}
\newcommand{\VerbBar}{|}
\newcommand{\VERB}{\Verb[commandchars=\\\{\}]}
\DefineVerbatimEnvironment{Highlighting}{Verbatim}{commandchars=\\\{\}}
% Add ',fontsize=\small' for more characters per line
\usepackage{framed}
\definecolor{shadecolor}{RGB}{248,248,248}
\newenvironment{Shaded}{\begin{snugshade}}{\end{snugshade}}
\newcommand{\AlertTok}[1]{\textcolor[rgb]{0.94,0.16,0.16}{#1}}
\newcommand{\AnnotationTok}[1]{\textcolor[rgb]{0.56,0.35,0.01}{\textbf{\textit{#1}}}}
\newcommand{\AttributeTok}[1]{\textcolor[rgb]{0.77,0.63,0.00}{#1}}
\newcommand{\BaseNTok}[1]{\textcolor[rgb]{0.00,0.00,0.81}{#1}}
\newcommand{\BuiltInTok}[1]{#1}
\newcommand{\CharTok}[1]{\textcolor[rgb]{0.31,0.60,0.02}{#1}}
\newcommand{\CommentTok}[1]{\textcolor[rgb]{0.56,0.35,0.01}{\textit{#1}}}
\newcommand{\CommentVarTok}[1]{\textcolor[rgb]{0.56,0.35,0.01}{\textbf{\textit{#1}}}}
\newcommand{\ConstantTok}[1]{\textcolor[rgb]{0.00,0.00,0.00}{#1}}
\newcommand{\ControlFlowTok}[1]{\textcolor[rgb]{0.13,0.29,0.53}{\textbf{#1}}}
\newcommand{\DataTypeTok}[1]{\textcolor[rgb]{0.13,0.29,0.53}{#1}}
\newcommand{\DecValTok}[1]{\textcolor[rgb]{0.00,0.00,0.81}{#1}}
\newcommand{\DocumentationTok}[1]{\textcolor[rgb]{0.56,0.35,0.01}{\textbf{\textit{#1}}}}
\newcommand{\ErrorTok}[1]{\textcolor[rgb]{0.64,0.00,0.00}{\textbf{#1}}}
\newcommand{\ExtensionTok}[1]{#1}
\newcommand{\FloatTok}[1]{\textcolor[rgb]{0.00,0.00,0.81}{#1}}
\newcommand{\FunctionTok}[1]{\textcolor[rgb]{0.00,0.00,0.00}{#1}}
\newcommand{\ImportTok}[1]{#1}
\newcommand{\InformationTok}[1]{\textcolor[rgb]{0.56,0.35,0.01}{\textbf{\textit{#1}}}}
\newcommand{\KeywordTok}[1]{\textcolor[rgb]{0.13,0.29,0.53}{\textbf{#1}}}
\newcommand{\NormalTok}[1]{#1}
\newcommand{\OperatorTok}[1]{\textcolor[rgb]{0.81,0.36,0.00}{\textbf{#1}}}
\newcommand{\OtherTok}[1]{\textcolor[rgb]{0.56,0.35,0.01}{#1}}
\newcommand{\PreprocessorTok}[1]{\textcolor[rgb]{0.56,0.35,0.01}{\textit{#1}}}
\newcommand{\RegionMarkerTok}[1]{#1}
\newcommand{\SpecialCharTok}[1]{\textcolor[rgb]{0.00,0.00,0.00}{#1}}
\newcommand{\SpecialStringTok}[1]{\textcolor[rgb]{0.31,0.60,0.02}{#1}}
\newcommand{\StringTok}[1]{\textcolor[rgb]{0.31,0.60,0.02}{#1}}
\newcommand{\VariableTok}[1]{\textcolor[rgb]{0.00,0.00,0.00}{#1}}
\newcommand{\VerbatimStringTok}[1]{\textcolor[rgb]{0.31,0.60,0.02}{#1}}
\newcommand{\WarningTok}[1]{\textcolor[rgb]{0.56,0.35,0.01}{\textbf{\textit{#1}}}}
\usepackage{graphicx}
\makeatletter
\def\maxwidth{\ifdim\Gin@nat@width>\linewidth\linewidth\else\Gin@nat@width\fi}
\def\maxheight{\ifdim\Gin@nat@height>\textheight\textheight\else\Gin@nat@height\fi}
\makeatother
% Scale images if necessary, so that they will not overflow the page
% margins by default, and it is still possible to overwrite the defaults
% using explicit options in \includegraphics[width, height, ...]{}
\setkeys{Gin}{width=\maxwidth,height=\maxheight,keepaspectratio}
% Set default figure placement to htbp
\makeatletter
\def\fps@figure{htbp}
\makeatother
\setlength{\emergencystretch}{3em} % prevent overfull lines
\providecommand{\tightlist}{%
  \setlength{\itemsep}{0pt}\setlength{\parskip}{0pt}}
\setcounter{secnumdepth}{-\maxdimen} % remove section numbering
\usepackage{graphicx}				% Use pdf, png, jpg, or eps§ with pdflatex; use eps in DVI mode
\usepackage{amssymb,amsmath}
\usepackage{natbib}
\usepackage{bm}
\usepackage{xcolor}
\usepackage{tikz}
\usepackage{ctable}
% \usepackage{newtxtext} % Times-like font
% \usepackage{titlesec}
% \titleformat*{\section}{\Large\bfseries\sffamily}
% \titleformat*{\subsection}{\large\bfseries\sffamily}

%%%%%%%%%%%
% Math Macros
\newcommand{\bmA}{\ensuremath{\bm A}}
\newcommand{\bma}{\ensuremath{\bm a}}
\newcommand{\bmB}{\ensuremath{\bm B}}
\newcommand{\bmb}{\ensuremath{\bm b}}
\newcommand{\bmC}{\ensuremath{\bm C}}
\newcommand{\bmc}{\ensuremath{\bm c}}
\newcommand{\bmD}{\ensuremath{\bm D}}
\newcommand{\bmd}{\ensuremath{\bm d}}
\newcommand{\bmE}{\ensuremath{\bm E}}
\newcommand{\bme}{\ensuremath{\bm e}}
\newcommand{\bmF}{\ensuremath{\bm F}}
\newcommand{\bmG}{\ensuremath{\bm G}}
\newcommand{\bmg}{\ensuremath{\bm g}}
\newcommand{\bmH}{\ensuremath{\bm H}}
\newcommand{\bmI}{\ensuremath{\bm I}}
\newcommand{\bmJ}{\ensuremath{\bm J}}
\newcommand{\bmK}{\ensuremath{\bm K}}
\newcommand{\bmL}{\ensuremath{\bm L}}
\newcommand{\bmM}{\ensuremath{\bm M}}
\newcommand{\bmP}{\ensuremath{\bm P}}
\newcommand{\bmQ}{\ensuremath{\bm Q}}
\newcommand{\bmq}{\ensuremath{\bm q}}
\newcommand{\bmR}{\ensuremath{\bm R}}
\newcommand{\bmr}{\ensuremath{\bm r}}
\newcommand{\bmS}{\ensuremath{\bm S}}
\newcommand{\bms}{\ensuremath{\bm s}}
\newcommand{\bmT}{\ensuremath{\bm T}}
\newcommand{\bmt}{\ensuremath{\bm t}}
\newcommand{\bmU}{\ensuremath{\bm U}}
\newcommand{\bmu}{\ensuremath{\bm u}}
\newcommand{\bmV}{\ensuremath{\bm V}}
\newcommand{\bmv}{\ensuremath{\bm v}}
\newcommand{\bmW}{\ensuremath{\bm W}}
\newcommand{\bmw}{\ensuremath{\bm w}}
\newcommand{\bmX}{\ensuremath{\bm X}}
\newcommand{\bmx}{\ensuremath{\bm x}}
\newcommand{\bmY}{\ensuremath{\bm Y}}
\newcommand{\bmy}{\ensuremath{\bm y}}
\newcommand{\bmZ}{\ensuremath{\bm Z}}
\newcommand{\bmz}{\ensuremath{\bm z}}


\newcommand{\bmalpha}{\ensuremath{\bm{\alpha}}}
\newcommand{\bmbeta}{\ensuremath{\bm{\beta}}}
\newcommand{\bmdelta}{\ensuremath{\bm{\delta}}}
\newcommand{\bmeta}{\ensuremath{\bm{\eta}}}
\newcommand{\bmepsilon}{\ensuremath{\bm{\epsilon}}}
\newcommand{\bmGamma}{\ensuremath{\bm{\Gamma}}}
\newcommand{\bmgamma}{\ensuremath{\bm{\gamma}}}
\newcommand{\bmLambda}{\ensuremath{\bm{\Lambda}}}
\newcommand{\bmmu}{\ensuremath{\bm{\mu}}}
\newcommand{\bmphi}{\ensuremath{\bm{\phi}}}
\newcommand{\bmSigma}{\ensuremath{\bm{\Sigma}}}
\newcommand{\bmtheta}{\ensuremath{\bm{\theta}}}
\newcommand{\bmzeta}{\ensuremath{\bm{\zeta}}}

\newcommand{\rank}{\ensuremath{\mathsf{rank}}}
\newcommand{\nullity}{\ensuremath{\mathsf{nullity}}}
\newcommand{\trace}{\ensuremath{\mathsf{tr}}}
\newcommand{\diag}{\ensuremath{\mathsf{diag}}}
\newcommand{\vecspan}{\ensuremath{\mathsf{span}}}

\newcommand{\mT}{\ensuremath{\mathsf{T}}}

\newcommand{\bbh}{\ensuremath{\hat{\bmbeta}}}
\newcommand{\Rn}{\ensuremath{\mathbb{R}^n}}
\newcommand{\Rnp}{\ensuremath{\mathbb{R}^{n\times p}}}
\newcommand{\Pa}{\ensuremath{{\bmP}_{\bmA}}}
\newcommand{\Pb}{\ensuremath{{\bmP}_{\bmB}}}
\newcommand{\Pv}{\ensuremath{{\bmP}_\mathcal{V}}}
\newcommand{\Pw}{\ensuremath{{\bmP}_\mathcal{W}}}
\newcommand{\Px}{\ensuremath{{\bmP}_{\bmX}}}
\newcommand{\XtX}{\ensuremath{\bmX^\mT\bmX}}
\newcommand{\XtXinv}{\ensuremath{(\bmX^\mT\bmX)^{-1}}}
\newcommand{\gtb}{\ensuremath{\bmg^\mT\bmbeta}}

\DeclareMathOperator*{\argmin}{arg\,min}
\newcommand{\bbmx}{\ensuremath{\begin{bmatrix}}}
\newcommand{\ebmx}{\ensuremath{\end{bmatrix}}}

\newcommand{\E}{\ensuremath{\mathrm{E}}}
\newcommand{\Var}{\ensuremath{\mathrm{Var}}}
\newcommand{\Cov}{\ensuremath{\mathrm{Cov}}}

\newcommand{\FWER}{\ensuremath{\mathrm{FWER}}}
\newcommand{\PCER}{\ensuremath{\mathrm{PCER}}}
\newcommand{\FDR}{\ensuremath{\mathrm{FDR}}}
\newcommand{\sFWER}{\ensuremath{\mathrm{sFWER}}}





\ifLuaTeX
  \usepackage{selnolig}  % disable illegal ligatures
\fi

\author{}
\date{\vspace{-2.5em}}

\begin{document}

\hypertarget{stat-640-homework-7}{%
\section{STAT 640: Homework 7}\label{stat-640-homework-7}}

Due \textbf{Wednesday, March 23, 11:59pm MT} on the course Canvas
webpage. Please follow the homework guidelines on the syllabus.

\hypertarget{name-hannah-butler}{%
\subsection{Name: Hannah Butler}\label{name-hannah-butler}}

\hypertarget{section}{%
\subsection{}\label{section}}

\hypertarget{problem-1}{%
\subsection{Problem 1}\label{problem-1}}

For this question, use the same data and model as Problem 4 on Homework
6. Consider the null hypothesis that there is no difference in chick
weight at age 6 between chicks on Diets 1, 2, and 4. (Note: this
hypothesis does not involve Diet 3.)

\textbf{a.} Provide the form of the linear model for weight as a
function of diet for the entire dataset of age 6 chicks. In other words,
copy your answer to Problem 4a from Homework 6.

\begin{center}\rule{0.5\linewidth}{0.5pt}\end{center}

\textbf{Answer:} \[\bmY = \bmX\bmbeta + \bmepsilon,\] where
\begin{align*}
\bmY_{49\times 1} &= 
\begin{bmatrix} 
Y_1 \\ 
\vdots \\ 
Y_49 
\end{bmatrix},
& \bmX_{49\times 4} &=
\begin{bmatrix}
\bm1_{19} & \bm0_{19} & \bm0_{19} & \bm0_{19} \\
\bm0_{10} & \bm1_{10} & \bm0_{10} & \bm0_{10} \\
\bm0_{10} & \bm0_{10} & \bm1_{10} & \bm0_{10} \\
\bm0_{10} & \bm0_{10} & \bm0_{10} & \bm1_{10}
\end{bmatrix},
& \bmbeta_{4\times 1} &= 
\begin{bmatrix}
\beta_1 \\
\beta_2 \\
\beta_3 \\
\beta_4
\end{bmatrix},
& \bmepsilon_{49\times 1} &=
\begin{bmatrix}
\epsilon_1 \\
\vdots \\
\epsilon_{49}
\end{bmatrix}
\end{align*} and since we will be testing a hypothesis on \(\bmbeta\),
we must make the distributional assumption \begin{align*}
\bmepsilon \sim N(\bm0, \sigma^2\bmI_{49\times 49})
\end{align*}

\begin{Shaded}
\begin{Highlighting}[]
\NormalTok{cw\_6 }\OtherTok{\textless{}{-}}\NormalTok{ ChickWeight }\SpecialCharTok{\%\textgreater{}\%}
  \FunctionTok{filter}\NormalTok{(Time }\SpecialCharTok{==} \DecValTok{6}\NormalTok{)}

\CommentTok{\# Response}
\NormalTok{Y }\OtherTok{\textless{}{-}}\NormalTok{ cw\_6}\SpecialCharTok{$}\NormalTok{weight}

\CommentTok{\# Design}
\NormalTok{X }\OtherTok{\textless{}{-}} \FunctionTok{cbind}\NormalTok{(}\FunctionTok{as.numeric}\NormalTok{(cw\_6}\SpecialCharTok{$}\NormalTok{Diet }\SpecialCharTok{==} \DecValTok{1}\NormalTok{)}
\NormalTok{           , }\FunctionTok{as.numeric}\NormalTok{(cw\_6}\SpecialCharTok{$}\NormalTok{Diet }\SpecialCharTok{==} \DecValTok{2}\NormalTok{)}
\NormalTok{           , }\FunctionTok{as.numeric}\NormalTok{(cw\_6}\SpecialCharTok{$}\NormalTok{Diet }\SpecialCharTok{==} \DecValTok{3}\NormalTok{)}
\NormalTok{           , }\FunctionTok{as.numeric}\NormalTok{(cw\_6}\SpecialCharTok{$}\NormalTok{Diet }\SpecialCharTok{==} \DecValTok{4}\NormalTok{)}
\NormalTok{           )}
\FunctionTok{colnames}\NormalTok{(X) }\OtherTok{\textless{}{-}} \FunctionTok{c}\NormalTok{(}\StringTok{"diet1"}
\NormalTok{                 , }\StringTok{"diet2"}
\NormalTok{                 , }\StringTok{"diet3"}
\NormalTok{                 , }\StringTok{"diet4"}
\NormalTok{                 )}
\end{Highlighting}
\end{Shaded}

\begin{center}\rule{0.5\linewidth}{0.5pt}\end{center}

\textbf{b.} Find \(\bmA\) such that \(\bmA\bmbeta = \bm0\) corresponds
to this hypothesis.

\begin{center}\rule{0.5\linewidth}{0.5pt}\end{center}

\textbf{Answer:}

\begin{align*}
\bmA &=
\begin{bmatrix}
1 & -1 & 0 & 0 \\
0 & 1 & 0 & -1 
\end{bmatrix}
\end{align*}

\(\bmA\bmbeta\) is testable because each row of \(\bmA\) is estimable
(ie, each row of \(\bmA\) is a linear combination of the rows of
\(\bmX\)).

Note that with \(\bmA\), we are testing the null hypothesis
\(H_0: \beta_1 = \beta_2\) AND \(\beta_2 = \beta_4\). If we fail to
reject \(H_0\), then this is evidence to suggest that there is no
difference between the 3 diets being tested. If we do reject \(H_0\),
then further testing would be required to determine for which diets
there is evidence of a difference. This is because there are three
situations in which we should reject \(H_0\); (1)
\(\beta_1 \neq \beta_2, \beta_2 = \beta_4\), (2)
\(\beta_1 = \beta_2, \beta_2 \neq \beta_4\), and (3)
\(\beta_1 \neq \beta_2, \beta_2 \neq \beta_4\). In the first two cases,
we can infer that one diet differs from the other two which do not
differ from each other, but in the third situation, it is not clear
whether \(\beta_1 = \beta_4\) or not and further testing is needed.

\begin{Shaded}
\begin{Highlighting}[]
\CommentTok{\# Hypothesis: No difference between diets 1, 2, and 4}
\NormalTok{A }\OtherTok{\textless{}{-}} \FunctionTok{rbind}\NormalTok{(}\FunctionTok{c}\NormalTok{(}\DecValTok{1}\NormalTok{, }\SpecialCharTok{{-}}\DecValTok{1}\NormalTok{, }\DecValTok{0}\NormalTok{, }\DecValTok{0}\NormalTok{)}
\NormalTok{           , }\FunctionTok{c}\NormalTok{(}\DecValTok{1}\NormalTok{, }\DecValTok{0}\NormalTok{, }\DecValTok{0}\NormalTok{, }\SpecialCharTok{{-}}\DecValTok{1}\NormalTok{)}
\NormalTok{           )}
\end{Highlighting}
\end{Shaded}

\begin{center}\rule{0.5\linewidth}{0.5pt}\end{center}

\textbf{c.} Compute \(RSS_H - RSS\) using only \(\bbh\), \(\bmA\), and
\(\bmX\).

\begin{center}\rule{0.5\linewidth}{0.5pt}\end{center}

\textbf{Answer:} By Proposition 5.4 in the notes, \emph{If
\(H:\bmA\bmbeta = \bm0\) is a testable hypothesis, then}
\[RSS_H - RSS = (\bmA\bbh)^T\left( \bmA(\bmX^T\bmX)^-\bmA^T\right)^{-1}(\bmA\bbh).\]
Where \(\bbh = (\bmX^T\bmX)^-\bmX^T\bmY\). In R:

\begin{Shaded}
\begin{Highlighting}[]
\CommentTok{\# (X\textquotesingle{}X)\^{}{-}}
\NormalTok{XX\_inv }\OtherTok{\textless{}{-}} \FunctionTok{solve}\NormalTok{(}\FunctionTok{t}\NormalTok{(X) }\SpecialCharTok{\%*\%}\NormalTok{ X)}

\CommentTok{\# Find beta estimate}
\NormalTok{bh }\OtherTok{\textless{}{-}}\NormalTok{ XX\_inv }\SpecialCharTok{\%*\%}\NormalTok{ (}\FunctionTok{t}\NormalTok{(X) }\SpecialCharTok{\%*\%}\NormalTok{ Y)}

\CommentTok{\# Compute difference in residual SS}
\NormalTok{RSS\_diff }\OtherTok{\textless{}{-}} \FunctionTok{t}\NormalTok{(A }\SpecialCharTok{\%*\%}\NormalTok{ bh) }\SpecialCharTok{\%*\%} \FunctionTok{solve}\NormalTok{( A }\SpecialCharTok{\%*\%}\NormalTok{ XX\_inv }\SpecialCharTok{\%*\%} \FunctionTok{t}\NormalTok{(A) ) }\SpecialCharTok{\%*\%}\NormalTok{ (A }\SpecialCharTok{\%*\%}\NormalTok{ bh)}
\NormalTok{RSS\_diff}
\end{Highlighting}
\end{Shaded}

\begin{verbatim}
##          [,1]
## [1,] 1972.773
\end{verbatim}

\begin{center}\rule{0.5\linewidth}{0.5pt}\end{center}

\textbf{d.} Conduct an F-test to test this hypothesis. Provide the test
statistic, p-value, and a conclusion statement.

\begin{center}\rule{0.5\linewidth}{0.5pt}\end{center}

\textbf{Answer:} Under the assumption of the null hypothesis
\(\bmA\bmbeta = \bm0\), we know, by Corollary 5.5.1, that the \(F\)
statistic is computed as \[F = \frac{(RSS_H - RSS)/q}{RSS/(n-r)},\] and
is distributed as \({\cal F}(q, n-r, 0)\). Using this, we can test
against and make a decision regarding \(H_0\).

\begin{Shaded}
\begin{Highlighting}[]
\NormalTok{n }\OtherTok{\textless{}{-}} \FunctionTok{length}\NormalTok{(Y)}
\NormalTok{r }\OtherTok{\textless{}{-}} \FunctionTok{rankMatrix}\NormalTok{(X)[}\DecValTok{1}\NormalTok{]}
\NormalTok{q }\OtherTok{\textless{}{-}} \FunctionTok{rankMatrix}\NormalTok{(A)[}\DecValTok{1}\NormalTok{]}

\NormalTok{RSS\_full }\OtherTok{\textless{}{-}} \FunctionTok{t}\NormalTok{(Y }\SpecialCharTok{{-}}\NormalTok{ (X }\SpecialCharTok{\%*\%}\NormalTok{ bh)) }\SpecialCharTok{\%*\%}\NormalTok{ (Y }\SpecialCharTok{{-}}\NormalTok{ (X }\SpecialCharTok{\%*\%}\NormalTok{ bh))}

\CommentTok{\# F statistic}
\NormalTok{fstat }\OtherTok{\textless{}{-}}\NormalTok{ (RSS\_diff}\SpecialCharTok{/}\NormalTok{q)}\SpecialCharTok{/}\NormalTok{(RSS\_full}\SpecialCharTok{/}\NormalTok{(n}\SpecialCharTok{{-}}\NormalTok{r))}
\FunctionTok{cat}\NormalTok{(fstat)}
\end{Highlighting}
\end{Shaded}

\begin{verbatim}
## 25.17208
\end{verbatim}

\begin{Shaded}
\begin{Highlighting}[]
\CommentTok{\# P{-}value}
\FunctionTok{cat}\NormalTok{(}\DecValTok{1} \SpecialCharTok{{-}} \FunctionTok{pf}\NormalTok{(fstat, q, n}\SpecialCharTok{{-}}\NormalTok{r))}
\end{Highlighting}
\end{Shaded}

\begin{verbatim}
## 4.604304e-08
\end{verbatim}

With an F statistic = 25.17208 and a \(p\)-value of
\(\approx 4.6 \times 10^{-8}\), we reject the null hypothesis that there
is no difference between the average effects of diets 1, 2, and 4 on the
weight of 6-day old chicks. In other words, there is evidence in the
data to suggest that at least one diet has a different effect on average
on the weight of 6-day old chicks.

\begin{center}\rule{0.5\linewidth}{0.5pt}\end{center}

\textbf{e.} Check your answer to (c) by fitting a model that corresponds
to the null hypothesis and calculating \(RSS_H\).

\begin{center}\rule{0.5\linewidth}{0.5pt}\end{center}

\textbf{Answer:}

Under the null hypothesis \(H_0: \beta_1 = \beta_2 = \beta_4\), we would
set \begin{align*}
\bmX_{H_0} &= 
\begin{bmatrix}
\bm1_{19} & \bm0_{19} \\
\bm1_{10} & \bm0_{10} \\
\bm0_{10} & \bm1_{10} \\
\bm1_{10} & \bm0_{10}
\end{bmatrix}
 & \text{ and } &&
\bmbeta &=
\begin{bmatrix}
\beta_{1,2,4} \\
\beta_3
\end{bmatrix}.
\end{align*}

\begin{Shaded}
\begin{Highlighting}[]
\CommentTok{\# C1 is diets 1, 2, or 4; C2 is diet 3}
\NormalTok{X\_H0 }\OtherTok{\textless{}{-}} \FunctionTok{cbind}\NormalTok{(X[,}\DecValTok{1}\NormalTok{] }\SpecialCharTok{+}\NormalTok{ X[,}\DecValTok{2}\NormalTok{] }\SpecialCharTok{+}\NormalTok{ X[,}\DecValTok{4}\NormalTok{]}
\NormalTok{              , X[,}\DecValTok{3}\NormalTok{]}
\NormalTok{              )}

\CommentTok{\# solve for b\_124 and b\_3}
\NormalTok{b\_H0 }\OtherTok{\textless{}{-}} \FunctionTok{solve}\NormalTok{( }\FunctionTok{t}\NormalTok{(X\_H0)}\SpecialCharTok{\%*\%}\NormalTok{X\_H0 ) }\SpecialCharTok{\%*\%} \FunctionTok{t}\NormalTok{(X\_H0)}\SpecialCharTok{\%*\%}\NormalTok{Y}

\CommentTok{\# Calculate estimated response under H0}
\NormalTok{Y\_H0 }\OtherTok{\textless{}{-}}\NormalTok{ X\_H0 }\SpecialCharTok{\%*\%}\NormalTok{ b\_H0}

\CommentTok{\# Calculate RSS under H0}
\NormalTok{RSS\_H0 }\OtherTok{\textless{}{-}} \FunctionTok{t}\NormalTok{(Y }\SpecialCharTok{{-}}\NormalTok{ Y\_H0) }\SpecialCharTok{\%*\%}\NormalTok{ (Y }\SpecialCharTok{{-}}\NormalTok{ Y\_H0)}

\CommentTok{\# Check difference}
\FunctionTok{cat}\NormalTok{(RSS\_H0 }\SpecialCharTok{{-}}\NormalTok{ RSS\_full)}
\end{Highlighting}
\end{Shaded}

\begin{verbatim}
## 1972.773
\end{verbatim}

\begin{center}\rule{0.5\linewidth}{0.5pt}\end{center}

\hypertarget{problem-2}{%
\section{Problem 2}\label{problem-2}}

Prove Proposition 5.11. That is, under the conditions of that
proposition, show that
\[\frac{RSS_H - RSS}{RSS} = \frac{R^2 - R^2_H}{1 - R^2}\]

\begin{center}\rule{0.5\linewidth}{0.5pt}\end{center}

\textbf{Answer:} \emph{(See Appendix for statement of Proposition 5.11)}

\begin{align*}
\frac{RSS_H - RSS}{RSS} &= \frac{\bmY^T(\bmP_{\bmX} - \bmP_{\bmX_H})\bmY}{\bmY^T(\bmI - \bmP_{\bmX})\bmY} \\
&= \frac{\bmY^T(\bmP_{\bmX} - \bmJ_n)\bmY - \bmY^T(\bmP_{\bmX_H} - \bmJ_n)\bmY}{\bmY^T(\bmI - \bmJ_n)\bmY - \bmY^T(\bmP_{\bmX} - \bmJ_n)\bmY} \\
&= \frac{RSS - RSS_H}{1 - \bmY^T(\bmP_{\bmX} - \bmJ_n)\bmY/\bmY^T(\bmI - \bmJ_n)\bmY}
\end{align*}

\begin{center}\rule{0.5\linewidth}{0.5pt}\end{center}

\hypertarget{problem-3}{%
\section{Problem 3}\label{problem-3}}

Consider the two regression lines
\[Y_{ki} = \beta_kx_i + \epsilon_{ki}\] for \(k=1, 2\) and
\(i=1, \dots, n\). Assume uncorrelated, homoscedastic errors. Find the
F-statistic for testing \(H : \beta_1 = \beta_2\).

\begin{center}\rule{0.5\linewidth}{0.5pt}\end{center}

\textbf{Answer:} This is a paired test. Consider the model \[
\bmY_{1} - \bmY_{2} = (\beta_1 - \beta_2)\bmx + (\bmepsilon_{1} - \bm\epsilon_{2}).
\] Let \(\alpha = \beta_1 - \beta_2\) and
\(\bmzeta = \bmepsilon_{1} - \bmepsilon_{2}.\) Then we have
\(\bmzeta = \bmepsilon_1 - \bmepsilon_2 \sim N(0, 2\sigma^2\bmI)\).

Now the null hypothesis can be reframed as \(H_0: \alpha = 0\) and for
an \(F\) test we can find \(RSS_H - RSS\) as \[
RSS_H - RSS = (\hat{\alpha} - 0 )^T(\bmx^T\bmx)^{-1}(\hat{\alpha} - 0) = \hat{\alpha}^2(\bmx^T\bmx)
\] and \(RSS\) as \[
RSS = ((\bmY_1 - \bmY_2) - \hat{\bmY})^T((\bmY_1 - \bmY_2) - \hat{\bmY}).
\] where \(\hat{\bmY} = E[\alpha\bmx + \bmzeta]\).

Since we are testing the value of 1 parameter, \(q=1\), and since we
have only one predictor, \(r = \rank(X) = 1\). So
\[F = \frac{\hat{\alpha}^2(\bmx^T\bmx)}{((\bmY_1 - \bmY_2) - \hat{\bmY})^T((\bmY_1 - \bmY_2) - \hat{\bmY})/(n-1)}.\]
and under the assumption of \(H_0: \alpha = 0\),
\(F \sim {\cal F}(1, n-1, 0)\).

\begin{center}\rule{0.5\linewidth}{0.5pt}\end{center}

\hypertarget{problem-4}{%
\section{Problem 4}\label{problem-4}}

Consider the linear model \(\bmY = \bmX\bmbeta + \bmepsilon\) with
\(\bmX \in \mathbb{R}^{n \times p}\) and \(\rank(\bmX) = r\). Let
\(\bmA\bmbeta = \bm0\) be a testable hypothesis with
\(\bmA \in \mathbb{R}^{q \times p}\) and \(q < r\). Prove that if
\(\rank(\bmA) = q\), then \(\rank(\bmA(\XtX)^{-}\bmA^\mT) = q\). (Hint:
recall that \(\rank(\bmB\bmB^\mT) = \rank(\bmB)\).)

\begin{center}\rule{0.5\linewidth}{0.5pt}\end{center}

\textbf{Answer:} We can first use the property
\(\rank(\bmA\bmB) \leq \min(\rank(\bmA), \rank(\bmB))\) to show that
\begin{align*}
\rank(\bmA(\bmX^T\bmX)^- \bmA^T) \le \min(\rank(\bmA), \rank((\bmX^T\bmX)^- \bmA^T)),
\end{align*} So \(\rank(\bmA(\bmX^T\bmX)^- \bmA^T) \le q\). Then by
Definition 5.1, we can write \(\bmA = \bmM\bmX\). So we have
\begin{align*}
\rank(\bmA(\bmX^T\bmX)^- \bmA^T) &= \rank(\bmM\bmX (\bmX^T\bmX)^- \bmX^T \bmM^T) \\
&= \rank(\bmM \bmP_{\bmX} \bmM^T) \\
&= \rank(\bmM \bmP_{\bmX} \bmP_{\bmX}^T\bmM^T) \\
&= \rank(\bmM\bmP_{\bmX}(\bmM\bmP_{\bmX})^T) \\
&= \rank(\bmM\bmP_{\bmX}).
\end{align*} Then, again using the property
\(\rank(\bmA\bmB) \leq \min(\rank(\bmA), \rank(\bmB))\), \begin{align*}
\rank(\bmA) &= \rank(\bmM\bmX) \\
&= \rank(\bmM\bmP_{\bmX}\bmX) \\
&\leq \min(\rank(\bmM\bmP_{\bmX}), \rank(\bmX))
\end{align*}

So that
\(\rank(\bmM\bmP_{\bmX}) = \rank(\bmA(\bmX^T\bmX)^- \bmA^T) \ge q\).
Therefore, \(\rank(\bmA(\bmX^T\bmX)^- \bmA^T) = q\).

\begin{center}\rule{0.5\linewidth}{0.5pt}\end{center}

\hypertarget{problem-5}{%
\section{Problem 5}\label{problem-5}}

Consider the linear model \begin{align*}
Y_1 &= \theta_1 + \theta_2 + \epsilon_1\\
Y_2 &= 2\theta_2 + \epsilon_2\\
Y_3 &= -\theta_1 + \theta_2 + \epsilon_3
\end{align*} where \(\E[\bmepsilon] = \bm0\) and
\(\Var(\bmepsilon) = \sigma^2\bmI\).

\textbf{a.} Show that \(H : \theta_1 = 2\theta_2\) is a testable
hypothesis.

\begin{center}\rule{0.5\linewidth}{0.5pt}\end{center}

\textbf{Answer:} \(H : \theta_1 - 2\theta_2\)

the vector \(\bmg^T = \begin{bmatrix} 1 & -2 \end{bmatrix}\) is
estimable (and hence \(\bma = \bmg^T\) is testable) because \[
\bmg^T = \begin{bmatrix} 0 & -1/2 & -1 \end{bmatrix} \begin{bmatrix} 1 & 1 \\ 0 & 2 \\ -1 & 1\end{bmatrix} = \begin{bmatrix} 1 & -2 \end{bmatrix},
\] where \(\begin{bmatrix} 1 & 1 \\ 0 & 2 \\ -1 & 1\end{bmatrix}\) is
the design matrix \(\bmX\).

\begin{center}\rule{0.5\linewidth}{0.5pt}\end{center}

\textbf{b.} Derive the form of the F-statistic for testing \(H\).

\begin{center}\rule{0.5\linewidth}{0.5pt}\end{center}

\textbf{Answer:}

First finding the components of the \(F\)-statistic, we have \[
(\bmX^T\bmX)^- = \begin{bmatrix} 2 & 0 \\ 0 & 6 \end{bmatrix}^{-1} = \begin{bmatrix} 1/2 & 0 \\ 0 & 1/6 \end{bmatrix}.
\] Then \[
\left(\bmA(\bmX^T\bmX)^- \bmA^T \right)^{-1} = \begin{bmatrix} 1 & -2 \end{bmatrix} \begin{bmatrix} 1/2 & 0 \\ 0 & 1/6 \end{bmatrix} \begin{bmatrix} 1 \\ -2 \end{bmatrix} = \frac{6}{7}
\] Computing \(\bbh\) next, we have \[
\bbh = (\bmX^T\bmX)^-\bmX^T\bmY = \begin{bmatrix} 1/2 & 0 \\ 0 & 1/6 \end{bmatrix} 
\begin{bmatrix} 1 & 0 & -1 \\ 1& 2 & 1\end{bmatrix}
\begin{bmatrix} Y_1 \\ Y_2 \\ Y_3\end{bmatrix}
= \begin{bmatrix} 1/2 & 0 \\ 0 & 1/6 \end{bmatrix} 
\begin{bmatrix} Y_1 - Y_3 \\ Y_1 + 2Y_2 + Y_3 \end{bmatrix}
= \begin{bmatrix} \frac{1}{2}(Y_1 - Y_3) \\ \frac{1}{6}(Y_1 + 2Y_2 + Y_3) \end{bmatrix}.
\] So \[
\bma\bbh = \begin{bmatrix} 1 & -2 \end{bmatrix}\begin{bmatrix} \frac{1}{2}(Y_1 - Y_3) \\ \frac{1}{6}(Y_1 + 2Y_2 + Y_3) \end{bmatrix} 
= \frac{1}{2}(Y_1 - Y_3) - \frac{1}{3}(Y_1 + 2Y_2 + Y_3)
= \frac{1}{6}(Y_1 - 4Y_2 - 5Y_3).
\] We can now compute \(RSS_H - RSS\): \[
RSS_H - RSS = \frac{6}{7}\cdot\frac{1}{36}(Y_1 - 4Y_2 - 5Y_3)^2 = \frac{1}{42} (Y_1 - 4Y_2 - 5Y_3)^2
\] To compute \(RSS\), we need to find \(\bmP_{\bmX}\): \begin{align*}
\bmP_{\bmX} &= \bmX(\bmX^T\bmX)^-\bmX \\
&= \frac{1}{3} 
\begin{bmatrix} 
2 & 1 & -1 \\
1 & 2 & 1 \\
-1 & 1 & 2
\end{bmatrix},
\end{align*} Then \begin{align*}
RSS &= \bmY^T(\bmI - \bmP_{\bmX})\bmY \\
&= \frac{1}{3}
\begin{bmatrix} Y_1 & Y_2 & Y_3 \end{bmatrix}
\begin{bmatrix}
1 & -1 & 1 \\
-1 & 1 & -1 \\
1 & -1 & 1
\end{bmatrix}
\begin{bmatrix} Y_1 \\ Y_2 \\ Y_3 \end{bmatrix} \\
&= \frac{1}{3}\left( Y_1^2 + Y_2^2 + Y_3^2 -2(Y_1Y_2 - Y_1Y_3 + Y_2Y_3) \right)
\end{align*} Then we get \[
F = \frac{\frac{1}{42} (Y_1 - 4Y_2 - 5Y_3)^2/1}{\frac{1}{3}\left( Y_1^2 + Y_2^2 + Y_3^2 -2(Y_1Y_2 - Y_1Y_3 + Y_2Y_3) \right)/(3-2)}
\]

\begin{center}\rule{0.5\linewidth}{0.5pt}\end{center}

\textbf{c.} If we assume the errors are normally distributed and the
null hypothesis is true, what is the distribution of \(F\)?

\begin{center}\rule{0.5\linewidth}{0.5pt}\end{center}

\textbf{Answer:} By Corollary 5.5.1, \(F \sim {\cal F}_{1,1,0}\) under
the assumption that \(H_0\) is true.

\begin{center}\rule{0.5\linewidth}{0.5pt}\end{center}

\newpage

\hypertarget{appendix}{%
\section{Appendix}\label{appendix}}

\hypertarget{list-of-r-packages-used}{%
\subsection{List of R packages used:}\label{list-of-r-packages-used}}

\begin{center}\rule{0.5\linewidth}{0.5pt}\end{center}

\texttt{tidyverse}

\texttt{ggplot2}

\texttt{Matrix}

\hypertarget{relevant-definitions}{%
\subsection{Relevant Definitions}\label{relevant-definitions}}

\begin{center}\rule{0.5\linewidth}{0.5pt}\end{center}

\textbf{Definition 5.1.} The hypothesis \(H: \bmA\bmbeta = \bm0\) is
\textbf{testable} if \(\bma_i^T\bmbeta\) is an estimable function, for
each row \(\bma_i^T\) of \(\bmA\).

\textbf{Definition 5.2.} The \textbf{residual sum of squares} for a
model is
\(RSS = \sum(Y_i-\hat{Y})^2 = (\bmY - \hat{\bmY})^T(\bmY - \hat{\bmY})\).

\textbf{Definition 5.4.} The \textbf{sample multiple correlation
coefficient} is the correlation between the pairs \((Y_i,\hat{Y_i})\).
That is: \[
R = \frac{\sum_i (Y_i - \bar{Y})(\hat{Y_i} - \bar{\hat{Y_i}})}{\sqrt{\sum_i (Y_i - \bar{Y})^2 \sum_i (\hat{Y_i} - \bar{\hat{Y_i}})^2}}
\]

\textbf{Definition 5.5.} The square of the sample multiple correlation
coefficient is called the \textbf{coefficient of determination}.

\hypertarget{relevant-propositions}{%
\subsection{Relevant Propositions}\label{relevant-propositions}}

\begin{center}\rule{0.5\linewidth}{0.5pt}\end{center}

\textbf{Proposition 5.3} \emph{Consider the two models
\(\bmY = \bmX_1\bmbeta_1 + \bmepsilon_1\) and
\(\bmY = \bmX_2\bmbeta_2 + \bmepsilon_1\). The difference in RSS between
these models is} \[
RSS_1 - RSS_2 = \bmY^T(\bmP_{\bmX_2} - \bmP_{bmX_1})\bmY
\]

\textbf{Proposition 5.4} \emph{If \(H: \bmA\bmbeta = \bm0\) is a
testable hypothesis, then}
\[RSS_H - RSS = (\bmA\bbh)^T\left(\bmA(\bmX^T\bmX)^- \bmA^T\right)^{-1}(\bmA\bbh).\]

\textbf{Proposition 5.5} \emph{If \(\bmY \sim N(\bmmu, \sigma^2\bmI)\)
with \(\bmmu = \bmX\bmbeta\) and \(H: \bmA\bmbeta = \bm0\) is a testable
hypothesis with \(\rank(\bmA) = q\), then:}
\[F = \frac{(RSS_H - RSS)/q}{RSS/(n-r)} \sim {\cal F}\left( q, n-r, \frac{1}{2\sigma^2}\bmmu^T(\bmP_{\bmX} - \bmP_{\cal W})\bmmu \right),\]
\emph{where \({\cal W} = {\cal N}(\bmM)\cap {\cal C}(\bmX)\) and
\(\bmM\) is a matrix such that \(\bmA = \bmM\bmX\).}

\textbf{Proposition 5.11} \emph{In the linear model
\(\bmY = \bmX\bmbeta + \bmepsilon\), assume \(\bmx_1 = \bm1\). Then the
test statistic for a hypothesis of the form
\(H: \begin{bmatrix} \bm0 & \bmA' \end{bmatrix} \bmbeta = \bm0\) (that
is, a hypothesis that does not involve the intercept \(\beta_1\)) can be
written:} \[F = \frac{R^2 - R^2_H}{1 - R^2} \frac{n-p}{q},\] \emph{where
\(R^2\) and \(R^2_H\) are the coefficients of determination for the full
and reduced model, respectively.}

\hypertarget{relevant-corollaries}{%
\subsection{Relevant Corollaries}\label{relevant-corollaries}}

\begin{center}\rule{0.5\linewidth}{0.5pt}\end{center}

\textbf{Corollary 5.5.1} \emph{If
\(\bmY sim N(\bmX\bmbeta, \sigma^2\bmI)\) and \(H: \bmA\bmbeta = \bm0\)
is a testable hypothesis with \(\rank(\bmA) = q\), then
\textbf{when $H$ is true},}
\[F = \frac{(RSS_H - RSS)/q}{RSS/(n-r)} \sim {\cal F}(q, n-r, 0).\]

\end{document}
